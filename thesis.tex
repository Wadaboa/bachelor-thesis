%--------------------------------------------------------------
% thesis.tex 
%--------------------------------------------------------------
% Corso di Laurea in Informatica 
% http://if.dsi.unifi.it/
% @Facolt\`a di Scienze Matematiche, Fisiche e Naturali
% @Universit\`a degli Studi di Firenze
%--------------------------------------------------------------
% - template for the main file of Informatica@Unifi Thesis 
% - based on Classic Thesis Style Copyright (C) 2008 
%   Andr\'e Miede http://www.miede.de   
%--------------------------------------------------------------
\documentclass[twoside,openright,titlepage,fleqn,
	headinclude,12pt,a4paper,BCOR5mm,footinclude]{scrbook}
%--------------------------------------------------------------
\newcommand{\myItalianTitle}{Libreria Python per l'OCR di documenti d'identit\`a\xspace}
\newcommand{\myEnglishTitle}{Python library to perform OCR on identity documents\xspace}
\newcommand{\myDegree}{Corso di Laurea in Informatica\xspace}
\newcommand{\myName}{Alessio Falai\xspace}
\newcommand{\myProf}{Maria Cecilia Verri\xspace}
\newcommand{\myOtherProf}{Correlatore\xspace}
\newcommand{\mySupervisor}{Michele Barbagli\xspace}
\newcommand{\myFaculty}{
	Scuola di Scienze Matematiche, Fisiche e Naturali\xspace}
\newcommand{\myUni}{\protect{
	Universit\`a degli Studi di Firenze}\xspace}
\newcommand{\myLocation}{Firenze\xspace}
\newcommand{\myTime}{Anno Accademico 2018-2019\xspace}
\newcommand{\myVersion}{Version 0.1\xspace}
%--------------------------------------------------------------
\usepackage[italian]{babel}
\usepackage[latin1]{inputenc} 
\usepackage[T1]{fontenc} 
\usepackage[square,numbers]{natbib} 
\usepackage[fleqn]{amsmath}  
\usepackage{ellipsis}
\usepackage{listings}
\usepackage{subfig}
\usepackage{caption}
\usepackage{appendix}
\usepackage{siunitx}
\usepackage{nameref}
\usepackage{tocstyle}
\usetocstyle{allwithdot}
%--------------------------------------------------------------
\usepackage{dia-classicthesis-ldpkg}
%--------------------------------------------------------------
% Options for classicthesis.sty:
% tocaligned eulerchapternumbers drafting linedheaders 
% listsseparated subfig nochapters beramono eulermath parts 
% minionpro pdfspacing
\usepackage[eulerchapternumbers,linedheaders,subfig,beramono,eulermath,
parts]{classicthesis}
%--------------------------------------------------------------
\newlength{\abcd} % for ab..z string length calculation
% how all the floats will be aligned
\newcommand{\myfloatalign}{\centering} 
\setlength{\extrarowheight}{3pt} % increase table row height
\captionsetup{format=hang,font=small}
%--------------------------------------------------------------
% Layout setting
%--------------------------------------------------------------
\usepackage{geometry}
\geometry{
	a4paper,
	ignoremp,
	textwidth     = 13.5cm,
	textheight    = 21.5cm,
	lmargin       = 3.5cm, % left margin
	tmargin       = 4cm    % top margin 
}

\lstset{
  	frame=tb,
	language=Matlab,
  	aboveskip=3mm,
  	belowskip=3mm,
  	showstringspaces=false,
  	columns=flexible,
  	basicstyle={\small\ttfamily},
  	numbers=none,
  	breaklines=true,
  	breakatwhitespace=true,
  	tabsize=3
}
%--------------------------------------------------------------
\begin{document}
\frenchspacing
\raggedbottom
\pagenumbering{roman}
\pagestyle{plain}
%--------------------------------------------------------------
% Frontmatter
%--------------------------------------------------------------
\include{title}
\pagestyle{scrheadings}
%--------------------------------------------------------------
% Mainmatter
%--------------------------------------------------------------
\pagenumbering{arabic}
% use \cleardoublepage here to avoid problems with pdfbookmark
%\chapter*{Introduzione}
\addcontentsline{toc}{chapter}{Introduzione}
\markboth{Introduzione}{Introduzione}

Questa tesi vuole essere un riassunto delle attivit\`a svolte in un tirocinio curriculare presso l'azienda QI-LAB di Firenze nel periodo \textit{marzo - giugno 2019}. L'offerta di tirocinio prevedeva l'apporto di migliorie, in termini di accuratezza ed efficienza, riguardanti una libreria per il riconoscimento ottico dei caratteri di alcuni campi di documenti d'identit\`a, attualmente solo italiani. La libreria, denominata QI-OCR, era inizialmente il risultato del lavoro compiuto da un precedente tirocinante, \textit{Emilio Cecchini}, che si era occupato dell'implementazione del framework di base, sul quale ho avuto il piacere di lavorare io stesso. QI-OCR viene distribuita come pacchetto \textit{pip} ed \`e stata scritta con linguaggio Python, in versione \textit{3.6.8}, anche se compatibile con la versione minor successiva, ovvero la \textit{3.7.x}, nonch\`e l'ultima versione stabile rilasciata, nella data in cui mi trovo a scrivere questo testo.\par
QI-OCR prevedeva la suddivisione in 4 moduli principali:
\begin{enumerate}
	\item \textbf{Preprocessing}: Si occupa della localizzazione del documento all'interno dell'immagine, del ritaglio dei contorni e del raddrizzamento, tramite l'algoritmo di feature matching \textit{SIFT}.
	\item \textbf{Docfields}: Si occupa di effettuare un ritaglio statico, mediante coordinate predeterminate, dei campi d'interesse del documento gi\`a centrato.
	\item \textbf{OCR}: Si occupa di effettuare una binarizzazione dei campi dell'immagine mediante un algoritmo di sogliatura globale \textit{ad-hoc} e di fornire le immagini prodotte in input a un motore di OCR.
	\item \textbf{Postprocessing}: Si occupa di applicare tecniche di analisi sintattica e semantica sui risultati ritornati dal software OCR.
\end{enumerate}
Il lavoro da me svolto riguarda il miglioramento della libreria QI-OCR, mediante tecniche che verranno descritte successivamente. A tal proposito, la tesi \`e suddivisa in cinque capitoli. Il capitolo \ref{chap:math-prerequisites} presenta una descrizione dei prerequisiti matematici necessari a comprendere correttamente le spiegazioni dei metodi presenti nei capitoli successivi. Il capitolo \ref{chap:image-binarization} riguarda lo studio e l'implementazione di tecniche di binarizzazione e segmentazione di immagini, che nel nostro caso vengono utilizzate per la "pulizia" di una generica immagine contenente testo, in modo da ottenere, idealmente, il solo testo di colore nero e tutto il resto di colore bianco. Il capitolo \ref{chap:image-matching} presenta una panoramica di algoritmi di \textit{image matching} e descrive metodi alternativi al ben noto SIFT per l'individuazione della posizione di un documento all'interno di un'immagine. Il capitolo \ref{chap:text-detection} introduce l'applicazione di tecniche avanzate di \textit{text-detection} con lo scopo di colmare problemi derivanti da documenti che non rispettano alcun tipo di template prefissato, come ad esempio la vecchia carta d'identit\`a cartacea italiana, che risulta comunque rilasciabile, in alternativa alla nuova carta d'identit\`a elettronica italiana, in casi di estrema esigenza o per tutti i cittadini che non possono recarsi per motivi di handicap presso il municipio. Infine, il capitolo \ref{chap:improvements} descrive ulteriori migliorie e innovazioni introdotte, sia per quanto riguarda il lato sviluppo/distribuzione (containerizzazione, benchmarks, \dots), sia per quanto riguarda l'esperienza utente (OCR parametrizzabile sui campi del documento, implementazione di diversi motori di OCR, \dots).
 % use \myChapter command instead of \chapter
\tableofcontents
\listoffigures
\cleardoublepage
\thispagestyle{empty}
\begin{flushright}
\null\vspace{\stretch {1}}
\emph{"If a machine is expected to be infallible, it cannot also be intelligent" \break --- Alan Turing} \vspace{\stretch{2}}\null
\end{flushright}
\cleardoublepage
\chapter*{Introduzione}
\addcontentsline{toc}{chapter}{Introduzione}
\markboth{Introduzione}{Introduzione}

Questa tesi vuole essere un riassunto delle attivit\`a svolte in un tirocinio curriculare presso l'azienda QI-LAB di Firenze nel periodo \textit{marzo - giugno 2019}. L'offerta di tirocinio prevedeva l'apporto di migliorie, in termini di accuratezza ed efficienza, riguardanti una libreria per il riconoscimento ottico dei caratteri di alcuni campi di documenti d'identit\`a, attualmente solo italiani. La libreria, denominata QI-OCR, era inizialmente il risultato del lavoro compiuto da un precedente tirocinante, \textit{Emilio Cecchini}, che si era occupato dell'implementazione del framework di base, sul quale ho avuto il piacere di lavorare io stesso. QI-OCR viene distribuita come pacchetto \textit{pip} ed \`e stata scritta con linguaggio Python, in versione \textit{3.6.8}, anche se compatibile con la versione minor successiva, ovvero la \textit{3.7.x}, nonch\`e l'ultima versione stabile rilasciata, nella data in cui mi trovo a scrivere questo testo.\par
QI-OCR prevedeva la suddivisione in 4 moduli principali:
\begin{enumerate}
	\item \textbf{Preprocessing}: Si occupa della localizzazione del documento all'interno dell'immagine, del ritaglio dei contorni e del raddrizzamento, tramite l'algoritmo di feature matching \textit{SIFT}.
	\item \textbf{Docfields}: Si occupa di effettuare un ritaglio statico, mediante coordinate predeterminate, dei campi d'interesse del documento gi\`a centrato.
	\item \textbf{OCR}: Si occupa di effettuare una binarizzazione dei campi dell'immagine mediante un algoritmo di sogliatura globale \textit{ad-hoc} e di fornire le immagini prodotte in input a un motore di OCR.
	\item \textbf{Postprocessing}: Si occupa di applicare tecniche di analisi sintattica e semantica sui risultati ritornati dal software OCR.
\end{enumerate}
Il lavoro da me svolto riguarda il miglioramento della libreria QI-OCR, mediante tecniche che verranno descritte successivamente. A tal proposito, la tesi \`e suddivisa in cinque capitoli. Il capitolo \ref{chap:math-prerequisites} presenta una descrizione dei prerequisiti matematici necessari a comprendere correttamente le spiegazioni dei metodi presenti nei capitoli successivi. Il capitolo \ref{chap:image-binarization} riguarda lo studio e l'implementazione di tecniche di binarizzazione e segmentazione di immagini, che nel nostro caso vengono utilizzate per la "pulizia" di una generica immagine contenente testo, in modo da ottenere, idealmente, il solo testo di colore nero e tutto il resto di colore bianco. Il capitolo \ref{chap:image-matching} presenta una panoramica di algoritmi di \textit{image matching} e descrive metodi alternativi al ben noto SIFT per l'individuazione della posizione di un documento all'interno di un'immagine. Il capitolo \ref{chap:text-detection} introduce l'applicazione di tecniche avanzate di \textit{text-detection} con lo scopo di colmare problemi derivanti da documenti che non rispettano alcun tipo di template prefissato, come ad esempio la vecchia carta d'identit\`a cartacea italiana, che risulta comunque rilasciabile, in alternativa alla nuova carta d'identit\`a elettronica italiana, in casi di estrema esigenza o per tutti i cittadini che non possono recarsi per motivi di handicap presso il municipio. Infine, il capitolo \ref{chap:improvements} descrive ulteriori migliorie e innovazioni introdotte, sia per quanto riguarda il lato sviluppo/distribuzione (containerizzazione, benchmarks, \dots), sia per quanto riguarda l'esperienza utente (OCR parametrizzabile sui campi del documento, implementazione di diversi motori di OCR, \dots).

\chapter{Binarizzazione di immagini}
\label{image-binarization}

\section{Introduzione}
\section{Sogliatura globale}
\section{Sogliatura adattativa (o locale)}
\section{Sogliatura di Otsu}
\section{Algoritmo di sogliatura proposto}

\chapter{Template matching}
\label{chap:template-matching}

\section{Introduzione}
\section{SIFT}
\section{SURF}
\section{AKAZE}
\section{BRISK}
\section{HED}

\chapter{Text detection}
\label{chap:text-detection}

\section{Introduzione}
\section{EAST}
\section{Algoritmo proposto}

\chapter{Ulteriori migliorie}
\label{improvements}

\section{Motori OCR}
\section{Benchmarks}
\section{Containerizzazione con Docker}
\section{Chiamate parametrizzabili sui campi del documento}
\section{Postprocessing}

\chapter*{Conclusioni e sviluppi futuri}
\addcontentsline{toc}{chapter}{Conclusioni e sviluppi futuri}
\markboth{Conclusioni e sviluppi futuri}{Conclusioni e sviluppi futuri}



\chapter*{Ringraziamenti}
\addcontentsline{toc}{chapter}{Ringraziamenti}
\markboth{Ringraziamenti}{Ringraziamenti}

\begin{thebibliography}{99}

\bibitem{bib:digital-image-processing}{Wilhelm Burger, Mark J. Burge - \emph{Digital Image Processing - An Algorithmic Introduction Using Java, Second Edition}}

\bibitem{bib:convolution}{Raffaele Cappelli - \emph{Fondamenti di Elaborazione di Immagini - Operazioni sulle immagini} - Universit\`a degli Studi di Bologna}

\bibitem{bib:wikipedia}{Wikipedia - \emph{\url{https://it.wikipedia.org}}}

\bibitem{bib:convolution}{Guocheng Wang, Yiwen Wang, Hui Li, Xuanqi Chen, Haitao Lu,
Yanpeng Ma, Chun Peng, Yijun Wang, Linyao Tang - \emph{Morphological Background Detection and Illumination Normalization of Text Image with Poor Lighting} - PLOS ONE}

\bibitem{bib:opencv}{OpenCV documentation - \emph{\url{https://docs.opencv.org}}}

\bibitem{bib:skin-thesis}{Simone Parca - \emph{Studio e sviluppo di algoritmi per l'analisi di immagini della pelle acquisite tramite un sensore capacitivo} - Universit\`a degli Studi di Bologna}

\bibitem{bib:binary-images-connectivity}{Francesco Tortorella - \emph{Elementi di geometria delle immagini digitali binarie} - Universit\`a degli studi di Cassino e del Lazio Meridionale}

\bibitem{bib:advanced-morphological}{\emph{Advanced morphological image processing} - University of Groningen}

\bibitem{bib:niblack}{W. Niblack - \emph{An introduction to Digital Image Processing} - Prentice-Hall}

\bibitem{bib:sauvola}{J. Sauvola, M. Pietikainen - \emph{Adaptive document image binarization} - Pattern Recognition 33(2), pp. 225-236}

\end{thebibliography}

%--------------------------------------------------------------
\end{document}
%--------------------------------------------------------------
