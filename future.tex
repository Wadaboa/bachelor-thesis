\chapter*{Conclusioni e sviluppi futuri}
\addcontentsline{toc}{chapter}{Conclusioni e sviluppi futuri}
\markboth{Conclusioni e sviluppi futuri}{Conclusioni e sviluppi futuri}

Nel campo dell'\textit{image processing} e del riconoscimento ottico dei caratteri (\textit{OCR}) risulta difficile produrre risultati accurati in modo efficiente. Lo studio degli algoritmi descritti in questa tesi ha permesso l'implementazione di svariate soluzioni per apportare miglioramenti su entrambi i lati della libreria QIOCR.\par
\begin{wrapfigure}{L}{0.35\textwidth}
	\centering
	\frame{\includegraphics[width=0.3\textwidth]{img/ts-r-example.png}}
	\caption{Tessera sanitaria retro con \textit{barcode}} \label{fig:ts-r-example}
\end{wrapfigure}
Con il lavoro svolto, siamo riusciti a incrementare notevolmente la percentuale di campi esattamente uguali ai valori direttamente osservabili, passando da un'accuratezza del 5\% a una pari a circa il 70\% sul numero totale di campi processati, che sono sono alcuni dei campi dei documenti d'identit\`a supportati. 
Per esempio, per la \textit{tessera sanitaria} viene considerato solamente il \textit{codice fiscale}, mentre per la \textit{carta d'identit\`a fronte} si considerano \textit{nome}, \textit{cognome} e \textit{data di nascita}. Nel caso di pi\`u campi il risultato passa il controllo solo se tutti rispettano l'uguaglianza con il dato effettivo.\par 
I tempi di esecuzione variano dai 5 ai 10 secondi per documento, a seconda della possibile difficolt\`a nelle operazioni di \textit{image matching}. In particolare, riusciamo a ottenere i risultati migliori nel caso in cui l'immagine analizzata sia una scansione, oppure quando questa \`e una fotografia scattata in modo parallelo rispetto al documento inquadrato.\par
\begin{wrapfigure}{R}{0.35\textwidth}
	\centering
	\frame{\includegraphics[width=0.3\textwidth]{img/passport.png}}
	\caption{Passaporto italiano con \textit{MRZ}\protect\footnotemark} \label{fig:passport}
\end{wrapfigure}
\footnotetext{Fonte: \url{http://www.esteri.it/mae/doc/specifiche_pass_italiani.pdf}}
Per quanto riguarda il futuro, si ipotizza l'implementazione di nuovi documenti, quali il passaporto italiano, e si prevede l'aggiunta di nuove funzioni, come la lettura di codici a barre e di \textit{MRZ} (\textit{Machine Readable Zones}). Inoltre, si considera il supporto di documenti d'identit\`a di altri paesi europei, in modo da poter estendere il mercato d'interesse del prodotto.\par
L'utilizzo della libreria \`e stato venduto ad alcuni clienti dell'azienda, che sfruttano tutt'ora le funzionalit\`a offerte.
