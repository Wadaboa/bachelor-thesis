\chapter{Binarizzazione di immagini}
\label{chap:image-binarization}


\section{Introduzione}
Nel campo dell'\textit{image processing} ricopre un ruolo fondamentale la possibilit\`a di distinguere diversi oggetti, forme e contorni presenti nell'immagine in analisi. Per far ci\`o \`e possibile ricorrere a svariate metodologie, che sono strettamente dipendenti dal contesto in cui si intende operare. Per esempio, nel caso del \textit{riconoscimento ottico dei caratteri} (OCR), le operazioni che andranno descritte assumono importanza assoluta, dato che questo tipo di applicazione, per operare al meglio, ha spesso bisogno di ricevere in input un'immagine in bianco e nero, in cui lo sfondo \`e di colore bianco e il testo \`e di colore nero. Gli argomenti affrontati, nel caso di un'applicazione OCR, saranno soprattutto utili nel caso in cui il motore utilizzato sia \textit{open-source} (vedi \textit{Tesseract}), ovvero nel caso in cui un buon \textit{pre-processing} dell'immagine possa fare la differenza. In questo articolo andremo quindi ad analizzare gli \textit{algoritmi} pi\`u ricorrenti nel campo del \textit{thresholding} di immagini e nella sezione \ref{sec:image-bin-proposed-approach} affronteremo un approccio specificamente studiato per i casi d'uso della libreria QI-OCR.

Da qui in avanti ipotizzeremo di lavorare con immagini originali in \textit{scala di grigi}, ovvero immagini che mantengono un singolo canale colore, anche se i metodi descritti possono facilmente essere generalizzati per immagini con pi\`u di un canale.


\section{Sogliatura globale}
\`E il pi\`u semplice metodo di \textit{thresholding}, che prevede la scelta di un determinato valore di soglia, compreso tra 0 e 255. L'algoritmo di sogliatura prevede quindi la scansione, pixel per pixel, dell'immagine e, nel caso in cui il valore del pixel esaminato sia minore del valore di soglia, tale pixel assumer\`a valore zero, che corrisponde al colore nero. Altrimenti, nel caso in cui il valore del pixel esaminato sia maggiore o uguale del valore di soglia, tale pixel assumer\`a valore 255, che corrisponde al colore bianco. Risulta facile intuire quale possa essere il problema principale di questo tipo di algoritmo di sogliatura, ovvero la scelta del valore di soglia.
Nel caso in cui il contesto in cui si opera sia relativamente statico, la \textit{sogliatura globale} risulta comunque l'opzione pi\`u consigliata, per la sua facilit\`a di implementazione. In particolare, il metodo descritto risulta adatto se le immagini in analisi presentano all'incirca caratteristiche uniformi in termini di illuminazione e contrasto. Se cos\`i non fosse, sarebbe necessario valutare l'implementazione di algoritmi leggermente pi\`u complessi.
\begin{algorithm}
	\caption{Sogliatura globale}
	\label{alg:thresh-global}
	\begin{algorithmic}[1]
		\Function{global-thresholding}{}
			\State $\vars{image} \gets \text{input image}$
			\State $\vars{thresh} \gets \text{threshold value}$
			
			\For {$i \text{ in \textproc{rows}(image)}$}
				\For {$j \text{ in \textproc{columns}(image)}$}
					\If {$\vars{image}[i][j] \geq \vars{thresh}$}
						\State $\vars{image}[i][j] \gets 255$
					\Else
						\State $\vars{image}[i][j] \gets 0$
					\EndIf
				\EndFor
			\EndFor

			\Return $\vars{image}$
		\EndFunction
	\end{algorithmic}
\end{algorithm}


\section{Sogliatura adattativa (o locale)}
Questo metodo di \textit{thresholding} consente di sopperire alle mancanze della sogliatura globale, nel caso in cui le immagini in analisi presentino illuminazione e/o contrasto non uniforme. In particolare, questa tecnica dinamica computa automaticamente differenti valori di soglia per diverse aree dell'immagine. Dunque, l'immagine viene suddivisa in tante \textit{sotto-immagini}, abbastanza piccole da poter ipotizzare che in ciascuna sotto-immagine illuminazione e contrasto siano sufficientemente uniformi. Una volta suddivisa l'immagine, un valore di soglia viene calcolato per ciascuna sotto-immagine. Il calcolo di ogni valore di soglia dipende dall'implementazione specifica dell'algoritmo, ma i metodi pi\`u utilizzati sono i seguenti:
\begin{itemize}
	\item Media dei valori dei pixel della sotto-immagine
	\item Mediana dei valori dei pixel della sotto-immagine
	\item Media fra valore massimo e minimo dei pixel della sotto-immagine
\end{itemize}
Per poter configurare al meglio questo tipo di operazione per il proprio contesto di applicazione, spesso \`e possibile impostare alcuni iperparametri. Nel caso specifico della libreria \textit{OpenCV}, \`e possibile selezionare un valore $c$, che viene sottratto da ciascun valore di soglia di ogni sotto-immagine. Per esempio, nel caso in cui il valore di soglia venga calcolato utilizzando la media $mean$, la soglia scelta non sar\`a esattamente $mean$, ma $mean - c$.


\section{Sogliatura di Otsu}
\section{Algoritmo di sogliatura proposto}
\label{sec:image-bin-proposed-approach}
