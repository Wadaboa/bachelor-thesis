\chapter*{Introduzione}
\addcontentsline{toc}{chapter}{Introduzione}

Questa tesi vuole essere un riassunto delle attivit\`a svolte in un tirocinio curriculare presso l'azienda QI-LAB di Firenze nel periodo \textit{marzo - giugno 2019}. L'offerta di tirocinio prevedeva l'apporto di migliorie, in termini di accuratezza ed efficienza, riguardanti una libreria per il riconoscimento ottico dei caratteri di alcuni campi di documenti d'identit\`a, attualmente solo italiani. La libreria, denominata QI-OCR, era inizialmente il risultato del lavoro compiuto da un precedente tirocinante, \textit{Emilio Cecchini}, che si era occupato dell'implementazione del framework di base, sul quale ho avuto il piacere di lavorare io stesso. QI-OCR viene distribuita come pacchetto \textit{pip} ed \`e stata scritta con linguaggio Python, in versione \textit{3.6.8}, anche se compatibile con la versione minor successiva, ovvero la \textit{3.7.x}, nonch\`e l'ultima versione stabile rilasciata, nella data in cui mi trovo a scrivere questo testo. QI-OCR prevedeva la suddivisione in 4 moduli principali:
\begin{enumerate}
	\item \textbf{Preprocessing}: Si occupa della localizzazione del documento all'interno dell'immagine, del ritaglio dei contorni e del raddrizzamento, tramite l'algoritmo di template matching \textit{SIFT}.
	\item \textbf{Docfields}: Si occupa di effettuare un ritaglio statico, mediante coordinate predeterminate, dei campi d'interesse del documento gi\`a centrato.
	\item \textbf{OCR}: Si occupa di effettuare una binarizzazione dei campi dell'immagine mediante un algoritmo di sogliatura globale \textit{ad-hoc} e di fornire le immagini prodotte in input a un motore di OCR.
	\item \textbf{Postprocessing}: Si occupa di applicare tecniche di analisi sintattica e semantica sui risultati ritornati dal software OCR.
\end{enumerate}
Il lavoro da me svolto riguarda il miglioramento della libreria QI-OCR, mediante tecniche che verranno descritte successivamente. A tal proposito, la tesi \`e suddivisa in cinque capitoli. Il capitolo \ref{chap:math-prerequisites} presenta una descrizione dei prerequisiti matematici necessari a comprendere correttamente le spiegazioni dei metodi presenti nei capitoli successivi. Il capitolo \ref{chap:image-binarization} riguarda lo studio e l'implementazione di tecniche di binarizzazione e segmentazione di immagini, ovvero soluzioni necessarie per la "pulizia" di una generica immagine contenente testo, che permettano di ottenere idealmente il solo testo di colore nero e tutto il resto di colore bianco. Il capitolo \ref{chap:template-matching} riguarda invece lo studio di algoritmi di segmentazione basata dal riscontro sul modello, in cui vengono introdotti metodi alternativi al ben noto SIFT per l'individuazione della posizione di un documento all'interno di un'immagine. Il capitolo \ref{chap:text-detection} introduce l'applicazione di tecniche avanzate di \textit{text-detection} con lo scopo di colmare problemi derivanti da documenti che non rispettano alcun tipo di template prefissato, come ad esempio la vecchia carta d'identit\`a cartacea italiana, che risulta comunque rilasciabile, in alternativa alla nuova carta d'identit\`a elettronica italiana, in casi di estrema esigenza o per tutti i cittadini che non possono recarsi per motivi di handicap presso il municipio. Infine, il capitolo \ref{chap:improvements} descrive ulteriori migliorie e innovazioni introdotte, sia per quanto riguarda il lato sviluppo/distribuzione (containerizzazione, benchmarks, \dots), sia per quanto riguarda l'esperienza utente (OCR parametrizzabile sui campi del documento, implementazione di diversi motori di OCR, \dots).
