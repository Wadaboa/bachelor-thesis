\chapter*{Introduzione}
\addcontentsline{toc}{chapter}{Introduzione}
\markboth{Introduzione}{Introduzione}

Questo lavoro di tesi vuole essere una descrizione delle attivit\`a svolte durante il tirocinio curriculare presso l'azienda QI-LAB di Firenze nel periodo \textit{marzo - giugno 2019}. Il progetto di tirocinio prevedeva l'apporto di migliorie, in termini di accuratezza ed efficienza, riguardanti una libreria per il riconoscimento ottico dei caratteri di alcuni campi di documenti d'identit\`a, attualmente solo italiani. La libreria originale, denominata QI-OCR, era il risultato del lavoro compiuto da un precedente tirocinante, \textit{Emilio Cecchini}, che si era occupato dell'implementazione del framework di base, sul quale ho avuto il piacere di lavorare io stesso.\par
QI-OCR \`e stata distribuita come pacchetto \textit{pip} ed \`e scritta in linguaggio Python, in versione \textit{3.6.8}, anche se compatibile con la versione minor successiva, ovvero la \textit{3.7.x}, nonch\'e l'ultima versione stabile rilasciata, nella data in cui mi trovo a scrivere questo testo. QI-OCR ha alcune dipendenze esterne, tra cui la libreria \textit{OpenCV}, per la manipolazione di immagini, e il software \textit{Tesseract}, per le funzionalit\`a di \textit{OCR}.\par
QI-OCR prevedeva la suddivisione in 4 moduli principali:
\begin{enumerate}
	\item \textbf{Preprocessing}: si occupa della localizzazione del documento all'interno dell'immagine, del ritaglio dei contorni e del raddrizzamento, tramite l'algoritmo di \textit{feature matching} \textit{SIFT}.
	\item \textbf{Docfields}: si occupa di effettuare un ritaglio statico, mediante coordinate predeterminate, dei campi d'interesse del documento gi\`a centrato.
	\item \textbf{OCR}: si occupa di effettuare una binarizzazione dei campi dell'immagine, mediante un algoritmo di sogliatura globale \textit{ad-hoc}, e di fornire le immagini prodotte in input a un motore di \textit{OCR}.
	\item \textbf{Postprocessing}: si occupa di applicare tecniche di analisi sintattica e semantica sui risultati restituiti dal software \textit{OCR}.
\end{enumerate}\par
Il lavoro da me svolto riguarda il miglioramento della libreria QI-OCR, mediante tecniche che verranno descritte successivamente. A tal proposito, la tesi \`e suddivisa in quattro capitoli. Il capitolo \ref{chap:math-prerequisites} presenta una descrizione dei prerequisiti matematici necessari a comprendere correttamente le spiegazioni dei metodi presenti nei capitoli successivi. Il capitolo \ref{chap:image-binarization} riguarda lo studio e l'implementazione di tecniche di binarizzazione e segmentazione di immagini, che nel nostro caso vengono utilizzate per la "pulizia" di una generica immagine contenente testo, in modo da ottenere, idealmente, il solo testo di colore nero e tutto il resto di colore bianco. Il capitolo \ref{chap:image-matching} presenta una panoramica di algoritmi di \textit{image matching} e descrive metodi alternativi al ben noto \textit{SIFT} per l'individuazione della posizione di un documento all'interno di un'immagine. Infine, il capitolo \ref{chap:text-detection} introduce l'applicazione di tecniche avanzate di \textit{text-detection} con lo scopo di colmare problemi derivanti da documenti che non rispettano alcun tipo di template prefissato, come ad esempio la vecchia carta d'identit\`a cartacea italiana, che risulta comunque rilasciabile, in alternativa alla nuova carta d'identit\`a elettronica italiana, in casi di estrema esigenza o per tutti i cittadini che non possono recarsi per motivi di handicap presso il municipio.\par 
In questa tesi vengono descritte le principali migliorie apportate alla libreria, anche se ne sono state introdotte di ulteriori, sia per quanto riguarda il lato sviluppo/distribuzione (containerizzazione con \textit{Docker}, \textit{benchmarks}, \dots), sia per quanto riguarda l'utilizzo finale del prodotto (\textit{OCR} parametrizzabile sui campi del documento, implementazione di diversi motori di \textit{OCR}, \dots).
